%%%%
% Преамбула: подключение необходимых пакетов
% Редактируйте осторожно!
%

\documentclass[hyperref={unicode}]{beamer}

\usepackage[utf8x]{inputenc}
\usepackage[english, russian]{babel}
\usepackage{color, colortbl}
\usepackage{rotating} 
\usepackage{graphicx}
\usepackage{algorithmic}

\usetheme[nosecheader]{PetrSU-CS}


%%%%
% Преамбула: основные параметры презентации
% Отредактируйте в соответствии с комментариями
%

\title[%
    % Краткое название работы не используется в этой презентации!
    Помощник-тренажер
]{%
    % Полное название работы отображается на титульной странице
    Помощник-тренажер по дисциплинам\\ 
    первого курса ПМиИ
}

% Подзаголовком опишите тип работы:
% - Курсовая работа
% - Выпускная квалификационная работа бакалавра
% - Дипломная работа
% - Магистерская диссертация
\subtitle{Промежуточный отчет о научно-исследовательской работе}

\author[%
    % Имя и фамилия автора работы отображаются на каждом слайде в нижнем колонтитуле
    Алексей Артамонов
]{%
    % Имя, отчество и фамилия автора работы отображаются на титульном слайде
    Алексей Романович Артамонов
}

\date[%
    % Дата защиты
    --.--.2024
]{%
    % Руководитель
    Научный руководитель: к.т.н., доцент Ю. А. Богоявленский
}

\institute[%
    % Краткое название организации не используется в этой презентации
    ПетрГУ
]{%
    % Полное название организации и подразделения
    Петрозаводский государственный университет\\
    Кафедра информатики и математического обеспечения
}


%%%%
%
% Начало содержимого слайдов
%

\begin{document}

% Титульный слайд
\begin{frame}
\maketitle
\end{frame}

% Пример слайда для обоснования актуальности работы
\begin{frame}
  % Заголовок слайда
  \frametitle{Неуспеваемость студентов}
  \framesubtitle{(обосновываем актуальность работы)}
   Студентам во время учебы необходимо успевать разбираться в нескольких новых дисциплинах сразу и в некоторых случаях восстанавливать пропуски в изученном материале.
  
  Возможности разбирать каждую тему с преподавателем нет, нужна альтернатива.
\end{frame}

% Пример слайда с формулировкой целей и задач
\begin{frame}
  % Заголовок слайда
  \frametitle{Цель и задачи}
  \framesubtitle{(формулируем цель работы и задачи для достижения цели)}
  \begin{block}{Цель работы}
    Решить проблемы неуспеваемости и плохого понимания студентами первого курса ПМиИ 
    нового материала по соответствующим дисциплинам этого курса.
  \end{block}
  \begin{block}{Задачи}
  \begin{itemize}
    \item Изучить новый язык программирования для разработки мобильных приложений;
    \item Изучить дисциплины и темы первого курса ПМиИ;
    \item Реализовать полученные знания в приложении-тренажере.
  \end{itemize}
  \end{block}
\end{frame}

\begin{frame}
  \frametitle{}
  
{\Large\mbox{}\hfil Спасибо за внимание!}
  
\end{frame}
\end{document}
